\begin{table}[H]
\begin{center}
\bgroup
\def\arraystretch{1.2}
\begin{tabular}{|c|c|c|c|}
\hline
\hline
\tit{Parameter} & \tit{Value} & \tit{Parameter} & \tit{Value}\\
\hline
\hline
$\mathsf{m}$ & $\in$ [1, 11] & $\tau$ & $\in$ [1, 41]\\ \hline
$\vass{T}{O}^{t = 0}$ & 0.0 & $\vass{T}{A}^{t = 0}$ & 0.0 \\ \hline
$\sigma^{t=0}_{\mathsf{O},\,\mathsf{for}}$ & 1.0 & $\sigma^{t=0}_{\mathsf{A},\,\mathsf{for}}$ & 2.0\\ \hline
$\sigma_{\mathsf{O},\,\mathsf{obs}}$ & 1.0 & $\sigma_{\mathsf{A},\,\mathsf{obs}}$ & 2.0 \\ \hline
$\sigma_\mathsf{F}$ & 1.0 & $\mathsf{N}_e$ & 25 \\
\hline
\hline
\end{tabular}
\egroup
\end{center}
\caption{\small Ranges and values of the described parameters. Note that the largest value of `$\mathsf{m}$' explored in our simulations is dictated by the cross-section of the stochastic forcing i.e. $\sigma_\mathsf{F}$; for $\sigma_\mathsf{F} = 1.0$, values of $\mathsf{m}\gtrsim 11$ result in significant (and dominant) dissipation in the model {$\mlin$} -- this has an adverse impact on the testability of {\enkf} due to the \tit{collapse} of the forecast ensemble.}
\label{table:table1}
\end{table}
